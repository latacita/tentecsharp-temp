%%==================================================================%%
%% Author : Abascal Fernández, Patricia                             %%
%%          Sánchez Barreiro, Pablo                                 %%
%% Version: 1.0, 11/06/2013                                         %%                                                                                    %%                                                                  %%
%% Memoria del Proyecto Fin de Carrera                              %%
%% Archivo raíz                                                     %%
%%==================================================================%%

\cdpchapter{Resumen}

\todo{Arreglar esto}

Dentro del Departamento de Matemáticas, Estadística y Computación se han desarrollado con anterioridad una serie de técnicas para la implementación y configuración de líneas de productos software para la plataforma .NET basándose en las clases parciales de C# [1]. Dichas técnicas se condensan en el denominado Slicer Pattern. Por otra parte, la metodología TENTE [2], desarrollada, entre otros, por el profesor Pablo Sánchez, define una serie de modelos y transformaciones de modelo a código para el desarrollo y configuración de líneas de productos software.

El objetivo de presente proyecto fin de carrera es desarrollar las herramientas necesarias para integrar las técnicas de implementación basadas en el Slicer Pattern dentro de la metodología TENTE. Para ello, el alumno deberá:

1. Implementar una serie de generadores de código que transformen modelos orientados a características en UML 2.0 a una implementación en C# basada en el Slicer Pattern.
2. Implementar una serie de generadores de código que permitan configurar una implementación de una línea de productos basada en el Slicer Pattern a partir de de una selección de características determinada.

Dichos generadores de código deberán integrarse como plugins dentro de la herramienta de modelado UML MagicDraw y de Eclipse. Los generadores de código se implementarán usando EGL [3], el lenguaje de transformación modelo a código de la suite de herramientas para la manipulación de modelos Epsilon.

\paragraph{Palabras Clave} \ \\

Línea de Productos Software, Generación de Código, Desarrollo Software Orientado a Características, TENTE, Clases Parciales C\#, Patrón Slicer, .NET, Epsilon.



 