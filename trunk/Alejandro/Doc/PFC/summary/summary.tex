%=============================================================================%
% Author : Alejandro P�rez Ruiz                                               %
% Author : Pablo S�nchez Barreiro                                             %
% Version: 1.0, 07/03/2011                                                    %
% Master Thesis: Resumen                                                      %
%=============================================================================%

\section{Summary}
The goal of a software product line \cite{pohl:2005} is to create an adequate infrastructure from which you could construct, as automatically as possible, specific products within a software products family. A software products family is a set of similar software applications, which therefore share some common characteristics, but also have variations between them.

The software to control a smart home is an example of a domain where product line approach is really adequate. This software provides a wide range of variations due to to different devices can be controlled in each home type (eg.: windows, doors, lights, heaters...) and the functions that you want those devices comply (eg.: presence simulation, automatic on/off lights, smart energy control...)

The aim of the project is a software product line for smart homes, so that you can automate the process of developing specific applications for specific homes. This infrastructure is developed on the platform .NET, using Visual Studio 2010. Partial classes of C\# are used as the main mechanism for modularization, composition and management of the different variable characteristics that compound the software product line. The description or informal specification of requirements to be met by the product line is based on an industrial case study provided by Siemens AG.

Thus, we have developed two plugins for development environment Visual Studio 2010, which the users could model smart homes, selecting the features that best fit their requirements. And through these models, the users could derivate automatically specific applications for each model.
