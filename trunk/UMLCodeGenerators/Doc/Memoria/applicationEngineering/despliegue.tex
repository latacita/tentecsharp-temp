%%==================================================================%%
%% Author : Abascal Fern�ndez, Patricia                             %%
%% Author : S�nchez Barreiro, Pablo                                 %%
%% Version: 1.5, 15/05/2013                                         %%
%%                                                                  %%
%% Memoria del Proyecto Fin de Carrera                              %%
%% Application Engineering/Despliegue                               %%
%%==================================================================%%

Una vez creados los generadores de c�digo, el siguiente paso es empaquetarlos y distribuirlos de forma que puedan ser usados de la forma m�s c�moda posible por diferentes desarrolladores para la creaci�n de productos concretos pertenecientes a nuestra familia de productos.

La forma m�s f�cil de distribuir nuestra infraestructura es crear un plugin,
que se integre con la plataforma Eclipse, que permita la generaci�n de un proyecto Visual Studio 2010 que contenga dos proyectos: (1) un proyecto que contiene todos los ficheros fuente para cada una de las caracter�sticas del modelo UML dado; y (2) otro proyecto que contiene los ficheros fuente espec�ficos para las caracter�sticas seleccionadas en la creaci�n de un producto espec�fico del modelo UML proporcionado. Esta secci�n describe el proceso
de creaci�n de dichas extensiones.

El �mbito de desarollo de plug-ins de Eclipse, \emph{Plug-in Development Environment} (PDE), proporciona un entorno c�modo para crear plug-ins e integrarlos con la plataforma Eclipse. De esta forma, realizamos el cambio de contexto necesario para poder invocar nuestro generador de c�digo principal, escrito en EGL, desde el lenguaje de programaci�n java que es el utilizado para la creaci�n de plug-ins en dicho entorno. Adem�s, debemos preprocesar el modelo UML de entrada y realizar ciertas modificaciones sobre el ya que algunas de sus directivas no son compatibles con dicho entorno. Una vez analizado el modelo de entrada procedemos a aplicar los generadores de c�digo de igual forma que venimos haciendo en las etapas anteriores del desarrollo del Proyecto. Por �ltimo, a�adimos el entorno gr�fico que nos permitir� acceder al plug-in c�modamente desde un men� integrado con Eclipse o desde un bot�n en la barra de tareas del mismo. A continuaci�n exportamos el proyecto como un plug-in y lo instalamos en nuestro sistema, de esta forma cuando iniciemos Eclipse podremos realizar la transformaci�n de modelo UML a c�digo C\# de forma sencilla.

Tanto los generadores de c�digo como la documentaci�n est�n disponibles a trav�s de una p�gina web, realizada con el �nico objetivo de dar a conocer el presente proyecto. Dicha web posee 5 secciones, las cuales se
describen a continuaci�n:
\textbf{Introducci�n:} Contiene un breve resumen acerca del �mbito y los objetivos del proyecto.
\textbf{Descargas:} Contiene los generadores de c�digo creados.
\textbf{Documentaci�n:} Contiene la documentaci�n de usuario y v�deos relacionados con el proyecto.
\textbf{Publicaciones:} Contendr� todas las publicaciones relacionadas con el proyecto.
\textbf{Contacto:} Datos de contacto con los participantes del proyecto.