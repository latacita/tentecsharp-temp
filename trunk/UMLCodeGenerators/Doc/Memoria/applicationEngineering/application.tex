%%==================================================================%%
%% Author : Abascal Fern�ndez, Patricia                             %%
%% Author : S�nchez Barreiro, Pablo                                 %%
%% Version: 1.2, 15/05/2013                                         %%
%%                                                                  %%
%% Memoria del Proyecto Fin de Carrera                              %%
%% Cap�tulo Application Engineering, Archivo ra�z                   %%
%%==================================================================%%

\chapterheader{Ingenier�a de Aplicaci�n}{Ingenier�a de Aplicaci�n}
\label{chap:application}

El cap�tulo anterior describi� el funcionamiento y desarrollo de los generadores de c�digo para la fase de \emph{Ingenier�a del Dominio}. Dichos generadores son los encargados de crear los esqueletos de la implementaci�n de referencia, la cual debe completarse manualmente. Este cap�tulo describe el funcionamiento y desarrollo de los generadores de c�digo para la fase de \emph{Ingenier�a de Aplicaciones}, los cuales tienen como objetivo adaptar la implementaci�n de referencia a las necesidades de cada cliente.

\chaptertoc

\section{Introducci�n}
\label{application:sec:intro}

%%==================================================================%%
%% Author : Abascal Fern�ndez, Patricia                             %%
%% Author : S�nchez Barreiro, Pablo                                 %%
%% Version: 1.2, 24/06/2013                                         %%
%%                                                                  %%
%% Memoria del Proyecto Fin de Carrera                              %%
%% Application Engineering/Introduccion                             %%
%%==================================================================%% 

El primer paso para crear un producto concreto, de acuerdo con la metodolog�a Te.Net (ver Secci�n~\ref{sec:intr:tenet}) es crear una selecci�n de aquellas caracter�sticas que se desea incluir en el producto. C�mo se crea dicha selecci�n de caracter�sticas est� fuera del �mbito de este proyecto fin de carrera. Referimos al lector interesado en tal asunto a otros proyectos fin de carrera presentados en esta misma Facultad sobre dicho tema~\citep{}.

Una vez que se tiene una selecci�n de caracter�sticas v�lida, utilizando dicha selecci�n de caracter�sticas, se configura la arquitectura de referencia creada en la fase de Ingenier�a del Dominio para crear un modelo arquitect�nico concreto, adaptado a las necesidades del cliente, del producto que queremos construir.  Dicho modelo arquitect�nico se obtiene de forma autom�tica mediante la utilizaci�n del lenguaje \emph{VML}~\citep{}, de acuerdo con la metodolog�a Te.Net (ver Secci�n~\ref{sec:intr:tenet}).





\section{Configuraci�n de productos a nivel arquitect�nico}
\label{application:sec:mod}

%%==================================================================%%
%% Author : Abascal Fernández, Patricia                             %%
%% Author : Sánchez Barreiro, Pablo                                 %%
%% Version: 1.0, 24/06/2013                                         %%
%%                                                                  %%
%% Memoria del Proyecto Fin de Carrera                              %%
%% Application Engineering/Modelo de Entrada                        %%
%%==================================================================%%

La estrategia para crear un modelo arquitectónico concreto consiste, tal como se describió en la Sección~\ref{}, en crear un paquete vacío, el cual representa el producto a construir, y añadir relaciones \emph{merge} a aquellas características que se desean incluir en el producto final. De esta forma, el contenido de los paquetes correspondiente a características que se deben incluir en el producto final, se combinan o componen en el paquete que representa el producto final.

Para distinguir el paquete que representa el producto final de los paquetes que representan características, se ha creado un perfil de UML 2.0~\citep{omg:uml:2005}. Un perfil UML 2.0 es un mecanismo genérico de extensión que permite personalizar los modelos UML para un propósito particular, mediante la especificación de estereotipos y valores etiquetados que modifican la semántica original de los elementos del modelo UML 2.0.

\begin{figure}[!tb]
  \center
  \includegraphics[width=\linewidth]{applicationEngineering/images/Configuracion1.eps} \\
  \caption{Configuración de un hogar inteligente completo}
  \label{app:fig:conf1}
\end{figure}

En nuestro caso, el perfil contiene un solo estereotipo, denominado \imp{SpecificProduct}, el cual se puede aplicar exclusivamente a paquetes UML tal como se puede apreciar en la Figura\ref{app:fig:conf1}. Además, por cada modelo UML 2.0 representando un producto concreto, sólo puede existir un paquete estereotipado de dicha forma. Esta última restricción se expresa por medio de OCL.

La Figura~\ref{app:fig:conf1} muestra un ejemplo de creación de un producto concreto dentro de la línea de productos software para hogares inteligentes. En este caso, se trata de un producto donde se ha incluido exclusivamente la característica de \imp{SmartEnergyMng}, lo que implica que deben seleccionarse además las características \imp{WindowMng} y \imp{HeaterMng}, ya que \imp{SmartEnergyMng} necesita que ambas características estén instaladas en un producto final para poder funcionar. Dicha dependencia queda especificada de forma explícita a través de las relaciones \emph{merge} existentes entre \imp{SmartEnergyMng} y \imp{WindowMng} y \imp{HeaterMng}. Debido a dichas relaciones, es imposible crear un producto que incluya \imp{SmartEnergyMng} pero no \imp{WindowMng} o \imp{HeaterMng}.

%\todo{Meter una figura con la configuración, hacedla en MagicDraw y que utilice el caso de estudio del proyecto}

La siguiente sección describe como este modelo arquitectónico puede transformarse automáticamente en el código necesario para crear una implementación concreta y completamente funcional de un producto software concreto.



\section{Algoritmo para la implementaci�n del producto espec�fico}
\label{application:sec:alg}

%%==================================================================%%
%% Author : Abascal Fern�ndez, Patricia                             %%
%% Author : S�nchez Barreiro, Pablo                                 %%
%% Version: 1.5, 15/05/2013                                         %%
%%                                                                  %%
%% Memoria del Proyecto Fin de Carrera                              %%
%% Application Engineering/Algoritmo                                %%
%%==================================================================%%

Para obtener una implementaci�n completamente funcional de un producto concreto, con unas caracter�sticas determinadas, de acuerdo con el \emph{Slicer Pattern} (ver Secci�n~\ref{sec:back:slicer}), es necesario: (1) crear una clase parcial por cada clase que deba estar incluida en el producto final; (2) crear la \emph{versi�n limpia} de cada constructor y cada m�todo que deba estar incluido en el producto final; y (3) hacer que dichas versiones limpias deleguen en las \emph{versiones sucias} que corresponda.

El primer paso en el proceso de transformaci�n es crear un nuevo proyecto y una nueva carpeta que represente el producto final.

Para calcular todas las clases que deben estar incluidas en el producto final, recorremos el modelo desde el paquete que representa el producto concreto, y que ser� siempre un paquete \emph{hoja}, hacia arriba, hasta llegar a la ra�z, o ra�ces, del modelo orientado a caracter�sticas. Normalmente, siempre hay un modelo ra�z que contiene los elementos que son comunes a todos los productos.
En nuestro caso de la Figura~\ref{app:fig:conf1}, dicho recorrido generar�a dos caminos distintos: (1) \imp{SmartEnergyMng}, \imp{WindowMng}, \imp{InitialModel}; y (2) \imp{SmartEnergyMng}, \imp{HeaterMng}, \imp{InitialModel}.

Obviamente, una clase puede aparecer en m�s de un paquete. Por ejemplo, la clase \imp{Gateway} aparece en todos los paquetes, a excepci�n del que representa el producto final, de la Figura~\ref{app:fig:conf1}. No obstante, cada clase que est� en un camino desde el paquete hoja al paquete ra�z, solo debe incluirse una vez en el producto final, aunque �sta aparezca varias veces. Por cada clase distinta presente en algunos de los caminos del paquete hoja a la ra�z, generamos una nueva clase, que colocamos en la carpeta que representa el producto final.

A continuaci�n, para cada clase, debemos calcular todos los m�todos limpios que debemos generar. Para ello, al igual que ocurr�a con las clases parciales, recorremos todos los caminos existentes de ra�z a hoja. Para cada clase, por cada m�todo distinto, es decir, con diferente signatura, creamos una versi�n limpia de dicho m�todo dentro de la clase parcial incluida en el producto final. El proceso de generaci�n del esqueleto del m�todo se realiza reutilizando las plantillas de generaci�n de c�digo y facilidades creadas para la Ingenier�a de Dominio.

Por �ltimo, quedar�a por generar el c�digo de cada m�todo, de forma que �ste delegue en la versi�n sucia del m�todo que corresponda. Es esta fase del algoritmo de generaci�n de c�digo la que entra�a mayor dificultad, porque pueden darse diversos casos. Analizamos cada caso a continuaci�n.

\subsection{Caso 1: S�lo existe una \emph{versi�n sucia} del m�todo}

Se trata del caso m�s simple. S�lo existe una \emph{versi�n sucia} del m�todo, por lo que hay que hacer es delegar en �l. En el ejemplo de la Figura~\ref{app:fig:conf1}, para la clase \imp{Gateway}, el m�todo \imp{WindowCtrl.open}  solo est� implementado en la caracter�stica \imp{WindowMng}, por lo que el c�digo generado para la \emph{versi�n limpia} de dicho m�todo simplemente contendr�a un delegado a la \emph{versi�n sucia} \imp{windowMng\_open} de dicho m�todo.

\subsection{Caso 2: Existen varias \emph{versiones sucias} independientes}

\begin{figure}[!tb]
  \center
  \includegraphics[width=0.60\linewidth,keepaspectratio=true]{applicationEngineering/images/Configuration(2).eps} \\
  \caption{Configuraci�n de una casa inteligente con versiones sucias independientes de un mismo m�todo}
  \label{app:fig:conf2}
\end{figure}

En este caso, existen varias \emph{versiones sucias} independientes del m�todo. Por independientes entendemos que dichas versiones se encuentran en caminos distintos, y ninguna es \emph{alcanzable} desde la otra. El ejemplo de la Figura~\ref{app:fig:conf1} no contiene ninguno de estos casos, por lo que usamos el ejemplo de la Figura~\ref{app:fig:conf2}, extra�do del mismo caso de estudio. Por razones de concisi�n y brevedad, en dicho ejemplo s�lo aparecen aquellos detalles que son relevantes para explicar la situaci�n que estamos tratando.

En este caso, se trata de una configuraci�n de un producto concreto que incluye las caracter�sticas \imp{BlindSimulation} y \imp{LightSimulation}, encargadas de simular la presencia de habitantes en el hogar mediante el movimiento de persianas y el encendido y apagado de luces. Obviamente, ambas caracter�sticas dependen de las caracter�sticas de gesti�n autom�tica de persianas (\imp{BlindMng}) y gesti�n autom�tica de luces (\imp{LightMng}), respectivamente. En cada una de estas caracter�sticas, se extiende la clase \imp{Gateway} para que contenga m�todos para iniciar y detener la simulaci�n (\imp{startSimulation} y \imp{stopSimulation}, respectivamente).

En este caso, la versi�n limpia de los m�todos \imp{startSimulation} y \imp{stopSimulation}, contenida dentro del paquete \imp{MyHome}, debe delegar en las versiones sucias del m�todo perteneciente tanto a \imp{BlindSimulation} como \imp{LightSimulation}, ya que en este caso, al inicial la simulaci�n de presencia, deben activarse tanto la simulaci�n de persianas como de luces. Es decir, por ejemplo, el m�todo \imp{startSimulation}, de \imp{MyHome}, contendr� en su interior llamadas a \imp{blindSimulation\_startSimulation} y a \imp{lightSimulation\_startSimualtion}. El orden el cual se generen estas llamadas es irrelevante.

\subsection{Caso 3: Existen \emph{versiones sucias} dependientes de un m�todo}

En este caso, existen varias \emph{versiones sucias} de un m�todo, pero dichas versiones est�n en el mismo camino, estando una situada a mayor profundidad, m�s cerca del paquete \emph{hoja} que la otra. Por ejemplo, en el caso de la Figura~\ref{app:fig:conf1}, existen dos versiones del m�todo \imp{openWindow}, de la clase \imp{Gateway}, en las caracter�sticas \imp{SmartEnergyMng} y \imp{WindowMng}. Ambas est�n en el mismo camino del paquete hoja al paquete ra�z (\imp{SmartEnergyMng}, \imp{WindowMng}, \imp{InitialModel}).

En este caso, de acuerdo a la sem�ntica del modelo UML 2.0, la versi�n del paquete \imp{SmartEnergyMng} debe sobrescribir la versi�n del paquete \imp{WindowMng}. Por tanto, la versi�n limpia del m�todo debe invocar en este caso s�lo a la versi�n sucia del paquete \imp{SmartEnergyMng}, ya que se entiende que esta versi�n \emph{m�s profunda} es la m�s actualizada. En caso de haber m�s de dos versiones dependientes, siempre se escoger�a la versi�n m�s profunda.

\subsection{Caso 4: Existen \emph{versiones sucias} dependientes e independientes de un m�todo}

Este caso se trata de una combinaci�n de los casos 2 y 3. Existen diversas versiones de un m�todo. Estas versiones las podemos agrupar en varios subconjuntos, donde cada subconjunto contiene todas las versiones que son dependientes entre s�. Por ejemplo, para la Figura~\ref{app:fig:conf1}, consideremos el caso del constructor de la clase \imp{Gateway}. Supongamos adem�s, que la caracter�stica \imp{LightMng} tambi�n est� seleccionada. Dicho constructor, aunque no se muestra de forma expl�cita en el diagrama, estar�a presente en todas las versiones de dicha clase, presente en cada una de las caracter�sticas del sistema.

Para la Figura~\ref{app:fig:conf1}, hay tres caminos distintos (recordemos que la caracter�stica \imp{LightMng} tambi�n est� seleccionada, aunque no aparezca en la figura): (1) \imp{MyHome}, \imp{SmartEnergyMng}, \imp{WindowMng}, \imp{InitialModel}; (2) \imp{MyHome}, \imp{SmartEnergyMng}, \imp{HeaterMng}, \imp{InitialModel}; y, (3) \imp{MyHome}, \imp{LightMng}, \imp{InitialModel}. En este caso, habr�a 5 versiones del constructor de la clase \emph{Gateway}, m�s concretamente \imp{smartEnergyMng\_Gateway}, \imp{windowMng\_Gateway}, \imp{heaterMng\_Gateway}, \imp{lightMng\_Gateway} y \imp{initialModel\_Gateway}. Tendr�amos dos conjuntos de m�todos dependientes, \{ \imp{smartEnergyMng\_Gateway}, \imp{windowMng\_Gateway}, \imp{heaterMng\_Ga-teway}, \imp{initialModel\_Gateway} \}, y \{ \imp{lightMng\_Gateway} y \imp{initialModel\_Gateway} \}.

En este caso, la versi�n limpia del m�todo debe invocar la versi�n m�s profunda de cada conjunto independiente de m�todos, en este caso \imp{smartEnergyMng\_Gateway} y \imp{lightMng\_Gateway}. Al igual que en el caso 2, el orden en el cual se invocan estos m�todos es irrelevante.

La siguiente secci�n describe, de forma muy superficial, c�mo se organizan las plantillas encargadas de implementar este no trivial algoritmo de generaci�n de c�digo.


\section{Generadores de C�digo C\#}
\label{application:sec:transf}
%%==================================================================%%
%% Author : Abascal Fern�ndez, Patricia                             %%
%% Author : S�nchez Barreiro, Pablo                                 %%
%% Version: 1.2, 15/05/2013                                         %%
%%                                                                  %%
%% Memoria del Proyecto Fin de Carrera                              %%
%% Application Engineering/Generadores de C�digo C#                 %%
%%==================================================================%%

Esta secci�n detalla la secuenciaci�n de las plantillas de generaci�n de c�digo creadas para implementar el algoritmo de la secci�n anterior. La Figura~\ref{app:fig:templates} muestra dicha secuenciaci�n.

\begin{figure}[!tb]
  \center
  \includegraphics[width=0.45\linewidth]{applicationEngineering/images/TemplatesAppEngineering.eps} \\
  \caption{Secuencia de ejecuci�n de las plantillas de generaci�n de c�digo (Ingenier�a de Aplicaciones)}
  \label{app:fig:templates}
\end{figure}

El punto de partida es id�ntico al utilizado para la fase de \emph{Ingenier�a del Dominio}; es decir, el generador de c�digo llamado \imp{ProjectCreation}, el cual crea el proyecto \emph{Visual Studio 2010} para el producto concreto. Este plantilla invoca a su vez a la plantilla \imp{SpecificProduct}, que es la que gobierna el proceso de generaci�n de c�digo a nivel de \emph{Ingenier�a de la Aplicaci�n}. Para ello, se invocan las dos siguientes plantillas principales.

La plantilla \imp{CalculateClassesAndCleanMethods} recorre todos los caminos existentes en el modelo de la arquitectura concreta, desde el paquete hoja a la ra�z, calculando las clases que hay que crear, las versiones limpias de m�todos a generar, as� como las versiones sucias en las cuales delegar, teniendo en cuenta los conceptos de independencia y alcanzabilidad.
A continuaci�n, se invoca la plantilla \imp{CleanVersionClassGeneration} para cada una de las clases calculadas anteriormente. Para cada clase calculada, se procede a generar todo el c�digo fuente necesario. Resaltar que estas clases, a diferencia de las generadas a nivel de \emph{Ingenier�a del Dominio}, s� son completamente ejecutables y contienen todo el c�digo necesario para ejecutar el producto completo.

Cada una de estas plantillas hace uso a su vez de otras subplantillas, que al igual que en la fase de \emph{Ingenier�a del Dominio}, se han obviado por razones de espacio y claridad.

Una vez implementados los generadores de c�digo para la fase de \emph{Ingenier�a de Aplicaci�n}, procedimos a realizar las pruebas pertinentes.


\section{Pruebas}
\label{application:sec:pruebas}
%%=======================================================================%%
%% Author : Abascal Fern�ndez, Patricia                                  %%
%% Author : S�nchez Barreiro, Pablo                                      %%   %%                                                                       %%
%% Version: 2.0, 25/06/2013                                              %%   %%                                                                       %%
%% Memoria del Proyecto Fin de Carrera                                   %% %% Domain Engineering/Pruebas con EUnit                                  %%   %%=======================================================================%%

Una vez implementados los generadores de c�digo, la siguiente tarea era comprobar su correcto funcionamiento. Para ello creamos, de forma sistem�tica, una serie de pruebas unitarias que permitiesen comprobar el correcto funcionamiento de los generadores de c�digo para un exhaustivo conjunto de diferentes tipos de entrada. Estas pruebas unitarias se implementaron en \emph{EUnit}~\cite{kolovos:2008}, el lenguaje de definici�n de pruebas de la suite Epsilon. El funcionamiento de \emph{EUnit} es similar al de \emph{JUnit}, pero aplicado a los lenguajes de la suite Epsilon, como EGL. 

Para comprobar que el funcionamiento del generador de c�digo es correcto, se dise�a el caso de prueba y se crea la salida de ese caso de prueba de forma manual. A continuaci�n, se ejecuta el caso de prueba y se comprueba que la salida generada coincide con la esperada, que es la creada manualmente. Al intentar implementar estas pruebas en \emph{EUnit}, nos encontramos con el problema inicial de que este lenguaje no ten�a implementada la comparaci�n de fragmentos de texto en ficheros, que era precisamente la funcionalidad que nos hac�a falta. No obstante, se curs� una petici�n a los creadores de Epsilon, los cuales, muy amablamente, incorporaron dicha funcionalidad a \emph{EUnit}. De igual forma, a�adieron otra funcionalidad para comprobar que dos directorios conten�an los mismos archivos (\imp{assertEqualDirectories}).

Para el dise�o de los casos de prueba se sigui� inicialmente un enfoque de caja negra, basado en una adaptaci�n de la t�cnica de clases de equivalencia y an�lisis de valores l�mite al entorno de los modelos software. Una vez ejecutadas estas pruebas, se analiz� la cobertura alcanzada, definiendo pruebas adicionales, ya de caja blanca, de forma que la cobertura alcanzada fuese del 100\%.
 
La Tabla~\ref{dom:table:prueba} resume algunos de los casos de prueba
ejecutados. Concretamente, se muestran los casos de prueba para analizar el comportamiento de las plantillas de generaci�n de c�digo para paquetes y clases. 

%%======================================================================%%
%%

\begin{table}%
\begin{small}
\begin{tabularx}{|l|p@10cm|l|}
 \hline
{}&{Casos v�lidos}&{Casos no v�lidos} \\ \hline
\multirow{4}{*}{Paquete} & Paquete con nombre. & Paquete sin nombre. \\
& Paquete con clases e interfaces en su interior. & \\
& Paquete vac�o. & \\
& Paquete dentro de otro paquete (recursividad). & \\
\hline
\multirow{12}{*}{Clase} & Clase con nombre. & Clase fuera de un paquete. \\
& Clase tipo abstract. & Clase sin nombre.\\
& Clase sin tipo. & \\
& Clase que hereda de una o varias clases. &\\
& Clase que hereda de una o varias interfaces. &\\
& Clase que hereda de clases e interfaces. &\\
& Clase sin propiedades. &\\
& Clase sin m�todos. &\\
& Clase sin propiedades ni m�todos. &\\
& Clase con propiedades. &\\
& Clase con m�todos. &\\
& Clase con propiedades y m�todos. &\\
\hline
\multirow{4}{*}{Clase Enumerada} & Clase enumerada con nombre. & Clase enumerada sin nombre. \\
& Clase enumerada con literales. & \\
& Clase enumerada vac�a. & \\
\hline
\multirow{3}{*}{Interfaz} & Interfaz con nombre. & Interfaz sin nombre. \\
& Interfaz sin m�todos. & Interfaz fuera de paquete.\\
& Interfaz con m�todos. & \\
\hline
\multirow{10}{*}{Propiedad} & Propiedad con nombre. & Propiedad sin tipo. \\
& Propiedad sin nombre (se debe poner uno por defecto). & Asociaciones sin multiplicidad.\\
& Propiedad est�tica (no lleva m�todos getter ni setter). & \\
& Propiedad protected (no lleva m�todos getter ni setter). & \\
& Propiedad no est�tica (lleva m�todos getter ni setter). & \\
& Propiedad es una colecci�n. & \\
& Propiedad es una asociaci�n simple. & \\
& Propiedad es una asociaci�n bidireccional one to one. & \\
& Propiedad es una asociaci�n bidireccional one to many. & \\
& Propiedad es una asociaci�n bidireccional many to many. & \\
\hline
\multirow{14}{*}{M�todo} & M�todo con nombre. &  M�todo sin nombre. \\
& M�todo sin tipo (se debe poner void por defecto). & \\
& M�todo sin tipo (se debe poner void por defecto) y sin par�metros. & \\
& M�todo sin tipo (se debe poner void por defecto) y con par�metros. &  \\
& M�todo void sin par�metros. &  \\
& M�todo void con par�metros. &  \\
& M�todo retorna tipo primitivo sin par�metros. &  \\
& M�todo retorna tipo primitivo con par�metros. &  \\
& M�todo retorna colecci�n sin par�metros. &  \\
& M�todo retorna colecci�n con par�metros. &  \\
& M�todo est�tico. &  \\
& M�todo abstracto. &  \\
& M�todo protected. &  \\
\hline
\multirow{3}{*}{Par�metros de m�todo} & Par�metro con nombre. & Par�metro sin tipo. \\
& Par�metro sin nombre (se debe poner uno por defecto). & \\
& Par�metro con tipo. & \\
\hline
\multirow{2}{*}{Herencia} & Herencia simple. &   \\
& Herencia m�ltiple (se debe implementar interfaces y clases adicionales, si fuera necesario). & \\
\hline
\end{tabularx}
\end{small}
\caption{Casos de prueba para lo generadores de Ingenier�a del Dominio}
\label{dom:table:prueba}
\end{table}%




\begin{lstlisting} [basicstyle=\ttfamily\scriptsize,language=CSharp, captionpos=b,
                    caption=Pruebas de los generadores de c�digo con EUnit,
                    label=dom:code:eunit]
01 @test
02 operation classWithNameAndWithoutType() {
03    assertLineWithMatch(path+"Data\\src\\BasicGraph\\Edge.cs",
                          "partial class Edge");
04 }
05 ...
06 @test
07 operation emptyPackage() {
08    assertEqualDirectories(path+"Data\\src\\PaqueteVacio",
                             path+"\\Data\\src\\PaqueteVacio");
09 }
10 ...
11 @test
12 operation thowsExceptions() {
13    ...
14    assertError(runTarget(pathTemplates+'\\ParametersCreation.egl'));
15    ...
16 }
\end{lstlisting}






\section{Sumario}

Durante este cap�tulo se han descrito la fase de \emph{Ingenier�a de Aplicaci�n} de nuestra l�nea de productos software. Dentro de dicha fase se ha analizado el algoritmo necesario para la generaci�n del producto espec�fico, se ha profundizado tambi�n en el desarrollo e implementaci�n de los generadores de c�digo necesarios para tal fin y se ha concluido con la fase de pruebas de dicha etapa del proyecto.

