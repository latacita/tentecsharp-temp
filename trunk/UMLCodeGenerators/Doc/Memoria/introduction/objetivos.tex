%%==================================================================%%
%% Author : Abascal Fern�ndez, Patricia                             %%
%%          S�nchez Barreiro, Pablo                                 %%
%% Version: 1.2, 11/06/2013                                         %%                                                                                    %%                                                                  %%
%% Memoria del Proyecto Fin de Carrera                              %%
%% Introduccion/Metodologia TeNet                                   %%
%%==================================================================%%

%%==================================================================%%
%% NOTA(Pablo): Te dejo tres p�rrafos, la idea es que los refundas  %%
%%              en uno y lo ligues con la secci�n anterior. Te      %%
%%              puedes extender un poco en detallar cada objetivo   %%
%%              si lo crees necesario, tal y como hizo Alejandro    %% 
%%==================================================================%%

El principal objetivo de este Proyecto de Fin de Carrera es implementar un conjunto de generadores de c�digo que permitan transformar modelos UML orientados a caracter�sticas en c�digo C\#. Para dar soporte a la orientaci�n a caracter�sticas a nivel de c�digo C\#, se utilizar� el patr�n de dise�o
\emph{Slicer}. Dicho patr�n fue espec�ficamente para tal prop�sito como parte de otro Proyecto Fin de Carrera presentado en esta misma Facultad~\cite{}.

%%%%%%%%%%%% Retoma el objetivo del proyecto %%%%%%%%%%%%
El objetivo de este Proyecto Fin de Carrera es implementar generadores de c�digo que abordar�n tanto la implementaci�n de la familia de productos software cubierta por la l�nea de productos, como la configuraci�n de productos concretos pertenecientes a dicha familia utilizando las prestaciones de las clases parciales en C\# y el Patr�n Slicer. Con esto esperamos haber aclarado el primer p�rrafo de esta secci�n al lector no familiarizado con las l�neas de productos software, clases parciales en lenguaje C\# y/o el Patr�n Slicer.


Tal como se ha descrito al inicio de este apartado, el objetivo del presente Proyecto Fin de Carrera consiste en el desarrollo e implementaci�n de unos generadores de c�digo que permitan la tranformaci�n del dise�o de los modelos en una implementaci�n en c�digo C\# de dichos dise�os, para ello se usar�n las prestaciones que ofrecen el uso de las clases parciales del lenguaje C\# basadas en el patr�n Slicer.
