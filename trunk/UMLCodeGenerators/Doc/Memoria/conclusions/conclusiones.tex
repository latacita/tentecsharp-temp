%%==================================================================%%
%% Author : Abascal Fern�ndez, Patricia                             %%
%% Author : S�nchez Barreiro, Pablo                                 %%
%% Version: 1.1, 27/06/2013                                         %%
%%                                                                  %%
%% Memoria del Proyecto Fin de Carrera                              %%
%% Conclusiones/Lecciones aprendidas                                %%
%%==================================================================%%

Esta secci�n describe las experiencias personales, buenas y malas, vividas a lo largo del proyecto. Para el desarrollo de los generadores de c�digo se utiliz� la herramienta Epsilon. Esta herramienta ofrece funcionalidades bastante potentes, las cuales facilitan bastante el trabajo de crear generadores de c�digo. No obstante, dada su reciente aparici�n y su constante desarrollo, a�n presentan ciertas carencias que dificultan su utilizaci�n.

Por ejemplo, uno de los mayores problemas que hemos tenido en la fase final de este proyecto, y que a�n estamos a la espera de resolver, ha sido en relaci�n a la integraci�n de las plantillas de generaci�n de c�digo con Eclipse. Por desgracia, nos ha  resultado imposible empaquetar los generadores de c�digo y ejecutarlos correctamente desde el plug-in creado, debido a una excepci�n interna producida por el propio Epsilon. Esta excepci�n estaba relacionado con un problema interno de Epsilon, que acaba siempre en un desbordamiento de memoria. Hemos reportado dicha incidencia a los creadores de Epsilon, estando actualmente a la espera de que la resuelvan.

En ciertos momentos del desarrollo de los generadores de c�digo, nos encontramos con ciertos errores en Epsilon, que los propios desarrolladores de Epsilon desconoc�an. Por ejemplo, a la hora de tratar modelos UML que tuvieran clases enumeradas se lanzaba una excepci�n que, tras reportarlo como bug y varios meses despu�s, fue solucionado por los desarrolladores en la �ltima versi�n de la herramienta.

Durante la fase de pruebas con \emph{EUnit}, encontramos que ciertas funcionalidades que eran necesarias para poder implementar los casos de prueba, como el poder comparar dos ficheros, no hab�an sido implementadas. Por tanto, nos pusimos en contacto con los desarrolladores de Epsilon para solicitar la inclusi�n de tales funcionalidades en EUnit. Con agradecimiento a estos desarrolladores, podemos decir, con cierto orgullo, que dichas peticiones fueron atendidas con cierta presteza.

Por �ltimo, comentar que EOL, el lenguaje que constituye el n�cleo de Epsilon, se trata de un lenguaje parecido a OCL~\citep{omg:uml:2005}, con un marcado car�cter funcional, el cual, al principio, resultaba complejo de utilizar. Adem�s, depurar el c�digo creado era un proceso tedioso y laborioso, ya que la depuraci�n deb�a realizarse mediante la mera inspecci�n de la salida producida.

No obstante, a pesar de estas dificultades, fue muy satisfactorio poder desarrollar los generadores de c�digo utilizando un lenguaje funcional, ya que difiere del tipo de lenguaje utilizado en los estudios de Ingenier�a Inform�tica de esta Facultad. Personalmente creo que la construcci�n de software orientado a caracter�sticas y las l�neas de producto software tienen bastante futuro, ya que permiten ahorrar tiempo, costes y crear productos acordes a las necesidades de cada usuario concreto. Por tanto, en cuanto las herramientas consigan estandarizarse, no me cabe duda que ser�n ampliamente utilizadas.
