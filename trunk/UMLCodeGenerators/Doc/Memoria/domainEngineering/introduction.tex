%%==================================================================%%
%% Author : Abascal Fern�ndez, Patricia                             %%
%% Author : S�nchez Barreiro, Pablo                                 %%
%% Version: 1.4, 21/06/2013                                         %%
%%                                                                  %%
%% Memoria del Proyecto Fin de Carrera                              %%
%% Domain Engineering/Transformaci�n UML a C#                       %%
%%==================================================================%% 

El primer paso a la hora de desarrollar un generador de c�digo es establecer una serie de correspondencias entre los distintos tipos de elementos que pueden aparecer en los modelos que sirven como entrada y los elementos del lenguaje de programaci�n destino. En nuestro caso, se trata de establecer una correspondencia entre elementos UML 2.0 y el lenguaje C\#, teniendo en cuenta que los elementos de entrada como los de salida deben seguir un enfoque orientado a caracter�sticas.


