%=============================================================================%
% Author : Alejandro P�rez Ruiz                                               %
% Author : Pablo S�nchez Barreiro                                             %
% Version: 1.1, 10/06/2011                                                    %
% Master Thesis: Background, master file                                      %
%=============================================================================%

\chapterheader{Antecedentes}{Antecedentes}
\label{chap:background}

Este cap�tulo proporciona una muy breve introducci�n a las diferentes t�cnicas y tecnolog�as que se usar�n a lo largo de este proyecto y que se entiende que no tienen por qu� ser conocidas por el lector. M�s concretamente, ofrece una visi�n general de las l�neas de productos software, el desarrollo de software orientado a caracter�sticas y las clases parciales de C\#. Recordemos que el objetivo de este proyecto es implementar una l�nea de productos usando clases parciales de C\# como mecanismo para encapsular y componer caracter�sticas.

\chaptertoc

\section{L�neas de Productos Software}

%%==================================================================%%
%% Author : Abascal Fern�ndez, Patricia                             %%
%%          S�nchez Barreiro, Pablo                                 %%
%% Version: 1.3, 18/06/2013                                         %%
%%                                                                  %%
%% Memoria del Proyecto Fin de Carrera                              %%
%% Background/Software Product Lines                                %%
%===================================================================%%

El objetivo de una \emph{l�nea de producto software}~\citep{pohl:2010,kakola:2006} es crear una infraestructura adecuada a partir de la cual se puedan derivar, de forma tan autom�tica como sea posible, producto concretos pertenecientes a una familia de producto software. Una familia de producto software es un conjunto de aplicaciones software similares, lo que implica que comparten una serie de caracter�sticas comunes, pero que tambi�n presentan variaciones entre ellos.

Un ejemplo cl�sico de familia de producto software es el producto Parten�n, para software bancario, comentado en la introducci�n a este documento (ver Secci�n~\ref{sec:intr:introduction}). Dicho producto representa una familia de productos destinados a la gesti�n bancaria. Parten�n en s� no puede ser desplegado como una aplicaci�n, sino que necesita ser configurado de acuerdo a una serie de caracter�sticas concretas demandadas por cada cliente que require una instalaci�n de Parten�n.

La idea de una l�nea de producto software es proporcionar una forma autom�tica y sistem�tica de construir productos concretos dentro de una familia de producto software mediante la simple especificaci�n de qu� caracter�sticas deseamos incluir dentro de dicho producto. Esto representa una alternativa al enfoque tradicional de desarrollo software, el cual se basaba simplemente en seleccionar el producto m�s parecido, dentro de la familia, al que queremos construir y adaptarlo manualmente.

El proceso de creaci�n de l�neas de producto software se estructura dos fases: (1) \emph{Ingenier�a del Dominio} (en ingl�s,  \emph{Domain Engineering}); y (2) \emph{Ingenier�a de Aplicaci�n} (en ingl�s, \emph{Application Engineering}) (ver Figura~\ref{back:fig:domainAplicEng}). La \emph{Ingenier�a del Dominio} tiene como objetivo la creaci�n de la infraestructura o arquitectura de referencia de la l�nea de productos software. Esta arquitectura de referencia debe permitir la r�pida, o incluso autom�tica, construcci�n de sistemas software espec�ficos pertenecientes a la familia de productos software. La \emph{Ingenier�a de Aplicaci�n} utiliza la infraestructura creada anteriormente para crear aplicaciones espec�ficas adaptadas a las necesidades de cada usuario en concreto.

\begin{figure}[!tb]
  \centering
  \includegraphics[width=.95\linewidth]{background/images/domainAplicationEngineering.eps} \\
  \caption{Proceso de Desarrollo de una l�nea de producto software}
  \label{back:fig:domainAplicEng}
\end{figure}

En la fase de Ingenier�a del Dominio, el primer paso a realizar es un an�lisis de qu� caracter�sticas de la familia de producto son variables y por qu� son variables. Esta parte es la que se conoce como \emph{An�lisis o Especificaci�n de la Variabilidad} (Figura~\ref{back:fig:domainAplicEng}, etiqueta 1).

A continuaci�n, se ha de dise�ar una arquitectura de referencia para la familia de producto software que permita soportar dicha variabilidad. Esta actividad se conoce como \emph{Realizaci�n o Dise�o de la Variabilidad} (Figura~\ref{back:fig:domainAplicEng}, etiqueta 2).

El siguiente paso es establecer una serie de reglas que especifiquen c�mo hay que instanciar o configurar la arquitectura previamente creada de acuerdo con las caracter�sticas seleccionadas por cada cliente. Esta fase es la que se conoce como \emph{Correspondencia entre Especificaci�n y Dise�o de la Variabilidad} (Figura~\ref{back:fig:domainAplicEng}, etiqueta 3).

Tras completar la fase de Ingenier�a del Dominio, disponemos de una especie de l�nea de montaje, la cual podemos utilizar para construir productos concretos de forma m�s o menos automatizada.

En la fase de Ingenier�a de Aplicaci�n, se crean productos concretos utilizando la infraestructura previamente creada. Para ello, el primer paso es crear una \emph{configuraci�n}, que no es m�s que una selecci�n de caracter�sticas que un usuario desea incluir en su producto concereto (Figura~\ref{back:fig:domainAplicEng}, etiqueta 4).

En el caso ideal, usando esta configuraci�n, se debe poder ejecutar las reglas de correspondencia entre especificaci�n y dise�o de la variabilidad para que la arquitectura creada en la fase de Ingenier�a del Dominio se adapte autom�ticamente; generando un producto concreto espec�fico acorde a las necesidades concretas del usuario (Figura~\ref{back:fig:domainAplicEng}, etiqueta 5). En el caso no ideal, dichas reglas de correspondencia deber�n ejecutarse a mano, lo cual suele ser un proceso tedioso, largo, repetitivo y propenso a errores.

La siguiente secci�n describe el paradigma de desarrollo software orientada a caracter�sticas, el cual est� �ntimamente ligado al dise�o e implementaci�n de l�neas de productos software.



\section{Ejemplo de LPS: El Problema de las Expresiones}

%=============================================================================%
% Author : Alejandro P�rez Ruiz                                               %
% Author : Pablo S�nchez Barreiro                                             %
% Version: 1.1, 10/06/2011                                                    %
% Master Thesis: Background/Expression Problem                                %
%=============================================================================%

El \emph{problema de las expresiones} es un problema t�pico dentro del mundo del dise�o software~\cite{cook:1990} y que es muy frecuentemente utilizado para ilustrar el funcionamiento de las diferentes t�cnicas y tecnolog�as relacionadas con las l�neas de productos software~\cite{lopezHerrejon:2004}. El objetivo del problema de las expresiones es dise�ar una familia de productos software que, para la gram�tica de la Figura~\ref{back:fig:gramExpr}, soporte siguientes operaciones:

\begin{description}
	\item[Print:] Debe mostrar por consola la expresi�n en el formato infijo, prefijo o posfijo.
	\item[Eval:] Debe evaluar la expresi�n y retornar su resultado.
	\item[ShortEval:] debe evaluar la expresi�n realizando las operaciones \emph{cortocircuitadas}. Es decir, tan pronto como el valor de un operando determine el resultado de la expresi�n, se deber� parar la evaluaci�n del resto de los operandos. Por ejemplo, en una multiplicaci�n, si el primer operando es 0, se retornar� el valor 0 directamente, sin evaluar el segundo operando.
\end{description}

No todas las operaciones tienen que aparecer en todos los productos, por lo que deber�a ser posible construir productos concretos que careciesen de alguna de ellas. 

\begin{figure}
\begin{center}
\begin{footnotesize}
\begin{verbatim}
Exp :: = Integer | AddInfix | MultInfix | AddPostfix | MulltPostfix |
				 AddPrefix | MultPrefix
Integer     :: <positive-negative integers>
AddInfix    ::= Exp "+" Exp
MultInfix   ::= Exp "*" Exp
AddPostfix  ::= Exp Exp "+"
MultPostfix ::= Exp Exp "*"
AddPrefix   ::= "+" Exp Exp
MultPrefix  ::= "*" Exp Exp
\end{verbatim}
\end{footnotesize}
\end{center}
\caption{Gram�tica del lenguaje de expresiones}
\label{back:fig:gramExpr}
\end{figure}

La siguiente secci�n describe c�mo usando �rboles de caracter�sticas podemos especificar la variabilidad existente en esta familia de productos software.



\section{�rboles de Caracter�sticas}
\label{sec:back:feature}

%=============================================================================%
% Author : Alejandro P�rez Ruiz                                               %
% Author : Pablo S�nchez Barreiro                                             %
% Version: 1.1, 10/06/2011                                                    %
% Master Thesis: Background/Feature Model                                     %
%=============================================================================%

Un �rbol de caracter�sticas~\cite{kang:1990,benavides:2010} es un �rbol and-or que se usa para especificar que elementos de una familia de productos (software)
son comunes a toda la familia de productos software, cuales son variables y por qu� dichos elementos son variables, por ejemplo, porque sean opcionales o alternativos entre s�.

%%===========================================================================%%
%% HECHO(Pablo): Cambia la ra�z de la siguiente figura para que en lugar de   %%
%%   EPL se llame ExpressionEditor                                           %%
%%   Poner una leyenda indicando que significa cada simbolo                  %%
%%   Hacer los Operators mandatory                                           %%    %%   Haz que la forma de imprimir sea alternativa                            %%
%%   Haz que los operadores sean alternativas mutuamente no exclusivas entre %%
%%    1 y 3                                                                  %%
%%   Las caracter�sticas Print_Infix, Print_Prefix y Print_Postfix, que se   %%
%%   llamen s�lo Infix, Prefix y Postfix.                                    %%
%%   Renombra Large Expression a Integer                                     %%
%%===========================================================================%%

\begin{figure}[!tb]
  \centering \includegraphics[width=.85\linewidth]{background/images/featureModelExpr.eps} \\
  \caption{�rbol de caracter�sticas para el \emph{problema de las expresiones}}
  \label{back:fig:featureModel}
\end{figure}

La Figura~\ref{back:fig:featureModel} muestra un �rbol de caracter�sticas para la familia de productos software para el problema de las expresiones descrito en la secci�n anterior. En dicho �rbol, los nodos representan las diferentes caracter�sticas de la familia de productos y las relaciones entre nodos especifican si dichas caracter�sticas son obligatorias, opcionales o alternativas (ver la notaci�n para cada caso en el cuadro \imp{Leyenda} de la Figura~\ref{back:fig:featureModel}).

M�s concretamente, el �rbol de la Figura~\ref{back:fig:featureModel} especifica que un \imp{ExpressionEditor} debe soportar obligatoriamente operaciones (\imp{Operations}), constantes \imp{Constants} y operadores \imp{Operators}.
Como operadores, pueden aparecer \imp{Add}, \imp{Mult} o \imp{LargeExpression}. Entre estos operadores se ha de escoger como m�nimo 1 y pueden seleccionarse los tres si as� se desea. Las operaciones \imp{Eval} y \imp{Print} son ambas opcionales. En el caso de \imp{Eval}, adem�s puede opcionalmente la opci�n de evaluaci�n con cortocircuito \imp{ShortEval}. En el caso de la operaci�n \imp{Print}, el usuario que adquiera un producto perteneciente a esta familia deber� escoger entre impresi�n prefija, infija o postfija.

No todas las relaciones y restricciones entre caracter�sticas se pueden
modelar usando la sintaxis propia de los �rboles de caracter�sticas. Cuando la expresividad de �stos no es suficiente, se suele modelar las relaciones entre caracter�sticas usando formulas de l�gica proposicional, donde los �tomos son los nodos del �rbol de caracter�sticas~\cite{batory:2005:propositional}. Por ejemplo, la evaluaci�n con cortocircuito s�lo tiene sentido si se ha seleccionado el operador \imp{Mult}, que es el �nico que permite evaluaci�n cortocircuitada. Esto se expresar�a mediante la f�rmula $ShortEval \Rightarrow Mult$.

Lo expuesto en esta secci�n cubrir�a la fase relativa a \emph{especificaci�n de la variabilidad}. El siguiente paso a la hora de construir una l�nea de productos software ser�a crear un dise�o software lo suficiente flexible para acomodar dichas variaciones. El caso ideal ser�a poder encapsular cada caracter�stica en un m�dulo software altamente cohesionado y d�bilmente acoplado cuya composici�n con otros m�dulos fuese adem�s lo m�s f�cil posible. De esta forma podr�amos construir productos software concretos dentro de esta familia de productos mediante la simple composici�n de los m�dulos correspondientes a las caracter�sticas deseadas por el usuario.

La siguiente secci�n explica c�mo se abordar�a este objetivo usando orientaci�n a objetos, los problemas que se plantean y c�mo surge la programaci�n orientada a caracter�sticas como soluci�n a dichos problemas.

%%=========================================================================%%
%% NOTA(Pablo): Esto se explica sobre la misma figura, eliminalo           %% %%=========================================================================%%
%%
%% Las relaciones entre las caracter�sticas padres y las caracter�sticas
%% hijas pueden ser clasificadas como:
%% \begin{enumerate}
%%    \item \emph{Obligatoria}: La caracter�stica hija es obligatoria.
%%    \item \emph{Opcional}: La caracter�stica hija es opcional.
%%    \item \emph{Simple}: La caracter�stica hija tendr� cardinalidad
%%            \emph{<m..n>}
%%    \item \emph{Grupo Or}: Al menos una de las caracter�sticas hijas debe ser
%%            seleccionada.
%%    \item \emph{Grupo Xor}: S�lo una de las caracter�sticas hijas debe ser
%%            seleccionada
%%    \item \emph{Grupo-Simple}: El n�mero de caracter�sticas seleccionadas del
%%            grupo vendr� dado por su cardinalidad.
%% \end{enumerate}
%%
%% Para representar visualmente las relaciones descritas anteriormente se
%% utiliza la notaci�n que se puede observar en la figura \ref{}.
%% \begin{figure}[!tb]
%%  \centering
%% \includegraphics[width=.65\linewidth]{background/images/notFeatureDiagram.eps} %% \\
%%  \caption{Notaciones utilizadas en los diagramas de caracter�sticas}
%%  \label{back:fig:fesatureModel}
%% \end{figure}
%%=========================================================================%%

%%===========================================================================%%
%% NOTA(Pablo): Este p�rrafo no lo entiendo, as� que lo suprimo              %% %%===========================================================================%%
%%
%% El objetivo de este tipo de diagramas a parte de la propia representaci�n y
%% facilidad para visualizar las diferentes caracter�sticas, es la posibilidad de %% utilizar las transformaciones de modelos para conseguir realizar
%% configuraciones que satisfagan las restricciones~\cite{czarnecki:2004}.
%%
%%===========================================================================%%

%%===========================================================================%%
%% NOTA(Pablo): He decidido que es mejor usar el problema de las expresiones %%
%%     para toda la secci�n                                                  %%    %%===========================================================================%%
%
% A modo de ejemplo se incluye en la figura \ref{back:fig:featurEShop} el
% diagrama de caracter�sticas que muestra un sistema configurable para una
% tienda online.
%
% \begin{figure}[!tb]
%  \centering
%  \includegraphics[width=.85\linewidth]{background/images/featureModelEshop.eps} %   \\
%   \caption{Diagrama de caracter�sticas para una \emph{e-shop}}
%   \label{back:fig:featurEShop}
% \end{figure}
%%===========================================================================%%

%%===========================================================================%%
%% NOTA(Pablo): Y este p�rrafo ha quedado ahora superfluo                    %% %%===========================================================================%%
%% Cuando se modelan l�neas de productos software a trav�s de los diagramas de
%% caracter�sticas,y se pretende generar el c�digo necesario para crear una
%% familia de productos software, los lenguajes de programaci�n orientados a
%% objetos resultan muchas veces insuficientes. Con objeto de paliar estas
%% deficiencias, en los �ltimos a�os han ido surgiendo unos nuevos tipos de
%% lenguajes denominados \emph{orientados a caracter�sticas}. La siguiente
%% secci�n introduce la programaci�n orientada a caracter�sticas para a
%% continuaci�n explicar sus ventajas con respecto a las t�cnicas tradicionales
%% orientadas a objetos.
%%===========================================================================%% 

\section{Programaci�n Orientada a Caracter�sticas}

%=========================================================================%
% Author : Alejandro P�rez Ruiz                                           %
% Author : Pablo S�nchez Barreiro                                         %
% Version: 1.1, 10/06/2011                                                %
% Master Thesis: Background/Feature Oriented Programming                  %
%=========================================================================%

La programaci�n orientada a caracter�sticas~\cite{prehofer:2001} surge como respuesta a las limitaciones que el dise�o y la programaci�n orientada a objetos posee en relaci�n a la implementaci�n de las l�nea de productos software. Por tanto, antes de describir en qu� consiste la programaci�n orientada a caracter�sticas, comentaremos qu� propiedades ser�a deseable encontrar en un lenguaje para la implementaci�n de l�neas de productos software y analizaremos c�mo satisfaces las t�cnicas de desarrollo software orientado a objetos dichas propiedades, comentando sus principales limitaciones.

\subsection{Propiedades deseables de un lenguaje de programaci�n para la implementaci�n de l�neas de productos software}

%%==================================================================%%
%% Author : Abascal Fern�ndez, Patricia                             %%
%%          S�nchez Barreiro, Pablo                                 %%
%% Version: 1.3, 18/06/2013                                         %%
%%                                                                  %%
%% Memoria del Proyecto Fin de Carrera                              %%
%% Background/Software Product Lines                                %%
%===================================================================%%

El objetivo de una \emph{l�nea de producto software}~\citep{pohl:2010,kakola:2006} es crear una infraestructura adecuada a partir de la cual se puedan derivar, de forma tan autom�tica como sea posible, producto concretos pertenecientes a una familia de producto software. Una familia de producto software es un conjunto de aplicaciones software similares, lo que implica que comparten una serie de caracter�sticas comunes, pero que tambi�n presentan variaciones entre ellos.

Un ejemplo cl�sico de familia de producto software es el producto Parten�n, para software bancario, comentado en la introducci�n a este documento (ver Secci�n~\ref{sec:intr:introduction}). Dicho producto representa una familia de productos destinados a la gesti�n bancaria. Parten�n en s� no puede ser desplegado como una aplicaci�n, sino que necesita ser configurado de acuerdo a una serie de caracter�sticas concretas demandadas por cada cliente que require una instalaci�n de Parten�n.

La idea de una l�nea de producto software es proporcionar una forma autom�tica y sistem�tica de construir productos concretos dentro de una familia de producto software mediante la simple especificaci�n de qu� caracter�sticas deseamos incluir dentro de dicho producto. Esto representa una alternativa al enfoque tradicional de desarrollo software, el cual se basaba simplemente en seleccionar el producto m�s parecido, dentro de la familia, al que queremos construir y adaptarlo manualmente.

El proceso de creaci�n de l�neas de producto software se estructura dos fases: (1) \emph{Ingenier�a del Dominio} (en ingl�s,  \emph{Domain Engineering}); y (2) \emph{Ingenier�a de Aplicaci�n} (en ingl�s, \emph{Application Engineering}) (ver Figura~\ref{back:fig:domainAplicEng}). La \emph{Ingenier�a del Dominio} tiene como objetivo la creaci�n de la infraestructura o arquitectura de referencia de la l�nea de productos software. Esta arquitectura de referencia debe permitir la r�pida, o incluso autom�tica, construcci�n de sistemas software espec�ficos pertenecientes a la familia de productos software. La \emph{Ingenier�a de Aplicaci�n} utiliza la infraestructura creada anteriormente para crear aplicaciones espec�ficas adaptadas a las necesidades de cada usuario en concreto.

\begin{figure}[!tb]
  \centering
  \includegraphics[width=.95\linewidth]{background/images/domainAplicationEngineering.eps} \\
  \caption{Proceso de Desarrollo de una l�nea de producto software}
  \label{back:fig:domainAplicEng}
\end{figure}

En la fase de Ingenier�a del Dominio, el primer paso a realizar es un an�lisis de qu� caracter�sticas de la familia de producto son variables y por qu� son variables. Esta parte es la que se conoce como \emph{An�lisis o Especificaci�n de la Variabilidad} (Figura~\ref{back:fig:domainAplicEng}, etiqueta 1).

A continuaci�n, se ha de dise�ar una arquitectura de referencia para la familia de producto software que permita soportar dicha variabilidad. Esta actividad se conoce como \emph{Realizaci�n o Dise�o de la Variabilidad} (Figura~\ref{back:fig:domainAplicEng}, etiqueta 2).

El siguiente paso es establecer una serie de reglas que especifiquen c�mo hay que instanciar o configurar la arquitectura previamente creada de acuerdo con las caracter�sticas seleccionadas por cada cliente. Esta fase es la que se conoce como \emph{Correspondencia entre Especificaci�n y Dise�o de la Variabilidad} (Figura~\ref{back:fig:domainAplicEng}, etiqueta 3).

Tras completar la fase de Ingenier�a del Dominio, disponemos de una especie de l�nea de montaje, la cual podemos utilizar para construir productos concretos de forma m�s o menos automatizada.

En la fase de Ingenier�a de Aplicaci�n, se crean productos concretos utilizando la infraestructura previamente creada. Para ello, el primer paso es crear una \emph{configuraci�n}, que no es m�s que una selecci�n de caracter�sticas que un usuario desea incluir en su producto concereto (Figura~\ref{back:fig:domainAplicEng}, etiqueta 4).

En el caso ideal, usando esta configuraci�n, se debe poder ejecutar las reglas de correspondencia entre especificaci�n y dise�o de la variabilidad para que la arquitectura creada en la fase de Ingenier�a del Dominio se adapte autom�ticamente; generando un producto concreto espec�fico acorde a las necesidades concretas del usuario (Figura~\ref{back:fig:domainAplicEng}, etiqueta 5). En el caso no ideal, dichas reglas de correspondencia deber�n ejecutarse a mano, lo cual suele ser un proceso tedioso, largo, repetitivo y propenso a errores.

La siguiente secci�n describe el paradigma de desarrollo software orientada a caracter�sticas, el cual est� �ntimamente ligado al dise�o e implementaci�n de l�neas de productos software.



\subsection{Limitaciones de la orientaci�n a objetos respecto a la implementaci�n de l�neas de productos software}
\label{back:sub:limitacionesOO}


\subsection{Lenguajes Orientados a Caracter�sticas}

Los lenguajes orientados a caracter�sticas~\cite{prehofer:2001} tienen como objetivo encapsular conjuntos coherentes de funcionalidad de un sistema software en m�dulos independientes y f�cilmente componibles de forma que se incremente la capacidad de reutilizaci�n y extensi�n de estos m�dulos. Dichos m�dulos reciben el nombre de \emph{caracter�stica}. Una caracter�stica se suele definir como un incremento de la funcionalidad de un sistema~\cite{batory:2004}.

%%===========================================================================%%
%% NOTA(Pablo): Si vemos que la memoria queda muy larga, aqu� se puede       %%
%%              recortar                                                     %%
%%===========================================================================%%

Los lenguajes orientados a caracter�sticas son especialmente �tiles en el contexto de las l�neas de productos software, ya que nos permiten encapsular en m�dulos bien definidos las diferentes caracter�sticas, tanto comunes como variables, que pueden aparecer en cada uno de los productos software pertenecientes a una misma familia. Los lenguajes orientados a caracter�sticas tratan de facilitar adem�s la composici�n de dichos m�dulos, contribuyendo as� a que productos espec�ficos dentro de una familia puedan ser creados mediante la simple composici�n o ensamblado de m�dulos software relativamente independientes.

%%============================================================================%%
%% NOTA(Pablo): Con la nueva reestructuraci�n de la secci�n, esta             %%
%%     argumentaci�n sobra                                                    %%
%%============================================================================%%
% Por ejemplo, a la hora de crear un tipo abstracto de datos pila, podemos
% considerar diferentes variaciones:
% \begin{description}
%     \item[B�sicas:] Toda pila, para ser considerada pila, debe soportar las
%            operaciones \imp{apilar} y \imp{desapilar}.
%     \item[Contador:] A�ade un contador para conocer el tama�o de la pila.
%     \item[Bloqueo:] Permite bloquear la pila para evitar modificaciones en su
%            estado.
%     \item[Deshacer:] Agrega la funcionalidad de restaurar el estado de la pila %            antes del �ltimo acceso a la misma.
% \end{description}
%
% El objetivo de la programaci�n orientada a caracter�sticas ser�a encapsular
% cada una de las cuatro funcionalidades anteriores en m�dulos independientes y
% f�cilmente componibles. De esta forma, se podr�an obtener diferentes productos % mediante la simple composici�n de conjuntos diferentes de caracter�sticas. Por % ejemplo, un determinado usuario podr�a estar interesado en una pila con
% contador, por lo que compondr�a estas dos caracter�sticas y descartar�a las
% dem�s. Otro usuario podr�a encontrar m�s adecuada una pila con bloqueo y
% deshacer, por lo que compondr�a estas tres caracter�sticas y dejar�a fuera la
% correspondiente al contador.
%
% Por lo tanto, un lenguaje orientado a caracter�sticas nos debe permitir
% descomponer f�cilmente un programa en caracter�sticas, las cuales deber�an
% encapsularse en m�dulos bien definidos y tan independientes como sea posible.
% A la hora de crear productos concretos, dichos m�dulos se compondr�an de
% acuerdo a las necesidades de los usuarios.
%%============================================================================%%

La programaci�n orientada a caracter�sticas
La siguiente secci�n describe en mayor detalle los problemas que surgen cuando se intenta obtener un enfoque orientado a caracter�sticas usando un lenguaje de programaci�n orientado a objetos.


\begin{figure}[!tb]
  \centering \includegraphics[width=.60\linewidth]{background/images/MixinPattern.eps} \\
  \caption{Diagrama de clases que hace uso del \emph{patr�n mixin}}
  \label{back:fig:mixin}
\end{figure}

Pero por fortuna o por desgracia, la herencia m�ltiple entre clases ha ido desapareciendo poco a poco en los lenguajes de programaci�n orientados a objetos actuales, tales como Java~\cite{arnold:2005} o C\#~\cite{albahari:2010}. Actualmente, la pr�ctica general es permitir que una clase pueda heredar de una sola clase y de un n�mero indeterminado de interfaces. Esto permite solventar los conocidos problemas de conflictos por colisi�n de comportamientos heredados, es decir, cuando un mismo m�todo con dos implementaciones distintas est� presente en clases padre diferentes. No obstante, en ausencia de este tipo de conflictos, la herencia m�ltiple puede ser un mecanismo muy �til~\cite{meyer:2000,meyer:2009}, por lo que puede que su simple erradicaci�n con el objetivo de evitar ciertas conflictos no sea una idea tan acertada como parece, cuya validez o no s�lo conoceremos con el paso del tiempo y la experiencia.

No obstante, la soluci�n para trabajar con herencia m�ltiple en lenguajes orientados a objetos con herencia simple, ser�a aplicar el \emph{patr�n mixin}~\cite{david:1986}. De acuerdo con dicho patr�n, que se ilustra en la figura \ref{back:fig:mixin}, se utilizan las interfaces para simular la herencia m�ltiple. Por lo tanto, cada clase a heredar implementar� una nueva interfaz que defina los m�todos de esta clase, a su vez la nueva clase que se quiere crear para que herede la funcionalidad de sendas clases implementar� las dos nuevas interfaces creadas anteriormente, y tendr� una relaci�n de agregaci�n con las clases a heredar.

Pero el uso del \emph{patr�n mixin} hace que, un incremento de funcionalidad representado por un conjunto de nuevas subclases que heredan de un conjunto de clases superiores, no sea posible manejarlo como un solo m�dulo. La insuficiente encapsulaci�n deriva en un incremento de la complejidad a la hora de configurar y construir nuevos productos por ensamblando o composici�n de caracter�sticas. A�n cuando las clases, separadas en paquetes, pertenezcan a una misma caracter�stica, es necesario seleccionar qu� clases concretas van a ser usadas en un producto espec�fico. Por ejemplo, para incluir la caracter�stica impresi�n infija para una configuraci�n del problema de las expresiones es necesario seleccionar todos los operadores binarios(\imp{AddInfix} y \imp{MultInfix}) y literales (\imp{IntegerInfix}) que implementan la caracter�stica.

Otro problema es el manejo de las dependencias, ya que la herencia tradicional obliga a que las clases y subclases tengan diferentes nombres. Por lo tanto, las referencias a las clases concretas deben ser actualizadas. A mayor n�mero de caracter�sticas en una l�nea de productos, las relaciones entre clases concretas se complican potencialmente, lo que resulta una situaci�n indeseable. Esto incrementa la complejidad en las relaciones y dependencias entre clases.Por ejemplo, en la caracter�stica impresi�n infija no ser�a permitido una clase \imp{Add} o \imp{Mult}. Se tienen que crear dos clases concretas, que hereden de \imp{Add} y \imp{Mult}, con diferentes nombres, \imp{AddInfix} y \imp{MultInfix}. Esto incrementa la complejidad en las relaciones y dependencias entre clases.

\subsubsection{Conclusiones}

Tras analizar el problema con un lenguaje orientado a objetos como es C\# se podr�a determinar que ser�a deseable encontrar un nuevo paradigma de  programaci�n, orientado a caracter�sticas,  donde se pudiesen obtener f�cilmente diferentes versiones de una misma aplicaci�n mediante la simple composici�n de caracter�sticas. Obviamente no todas las combinaciones de caracter�sticas ser�an v�lidas, por ejemplo para una caracter�stica de una aplicaci�n para un hogar inteligente que se encargue de realizar un uso eficiente del consumo de energ�a a trav�s del control de las perdidas por ventanas abiertas, necesitar� que las caracter�sticas relacionadas con la calefacci�n y las ventanas est�n seleccionadas. Por tanto, un lenguaje orientado a caracter�sticas deber�a asegurarse de que el resultado de la composici�n de un conjunto de caracter�sticas da lugar a aplicaciones v�lidas y consistentes.


\subsection{Ventajas de los lenguajes orientados a caracter�sticas}
Los lenguajes orientados a caracter�sticas nos otorgan una mayor flexibilidad, ya que nos permiten que clases individuales puedan ser compuestas por un conjunto de caracter�sticas, por tanto son especialmente recomendados para utilizarlos con las l�neas de producci�n software.

Para estudiar las ventajas de los lenguajes orientados a caracter�sticas se trabajar� con CaesarJ \cite{aracic:2006} que es un lenguaje de programaci�n basado en Java, que nos proporciona una mayor modularidad y el desarrollo a trav�s de componentes reusables. Para ello trabaja con conceptos como clases y paquetes en una �nica entidad, llamada familia de clases, que constituye unidades de encapsulamiento adicional para agrupar clases relacionadas. Una familia de clases tambi�n es, en s� misma una clase. As� mismo, se introduce el concepto de clases virtuales, que son clases internas (de familias de clases) propensas a ser refinadas a nivel de subclases. En el refinamiento de una clase virtual, impl�citamente se hereda de la clase que refina, por lo que tambi�n esto es visto como una relaci�n adjunta. Tambi�n, en un refinamiento pueden ser a�adidos nuevos m�todos, campos, relaciones de herencia y sobreescritura de m�todos. Puesto que en cada familia de clases, las referencias a las clases virtuales siempre apuntar�n al refinamiento m�s espec�fico. Esto significa que, por medio de las clases virtuales se aplica sobre escritura de m�todos, permitiendo redefinir el comportamiento de cualquier subclase de una familia de clases.

En t�rminos de programaci�n orientada a caracter�sticas, cada funcionalidad es modelada como una familia de clases. Mientras que los componentes y objetos del dominio espec�fico, son correspondidos por sus clases virtuales. As� mismo, las clases virtuales pueden ser declaradas como clases abstractas, lo que habilita la definici�n de interfaces en la implementaci�n modular de caracter�sticas.

Para hacer uso de lo citado anteriormente y ver su beneficio con las l�neas de producci�n software se ha vuelto a utilizar el ya comentado problema de las expresiones de la subsecci�n anterior. Pero en este caso, el dise�o cambia significativamente debido a la caracter�sticas expuestas de CaesarJ.

Por un lado en la figura \ref{back:fig:caesarJExpressions} se muestra que por cada operaci�n, tenemos una familia de clases, siendo la familia de clases que encapsula a todas las dem�s, la denominada \imp{Expressions}. �sta tiene la estructura de clases representado en la figura \ref{back:fig:expr}, y cada familia de clases lo que hace es redefinir las clases virtuales de \imp{Expressions} con las operaciones necesarias en cada caso.

\begin{figure}[ht!]
  % Requires \usepackage{graphicx}
  \centering \includegraphics[width=.60\linewidth]{background/images/CaesarJExpressions.eps} \\
  \caption{Dise�o para resolver el problema de las expresiones con CaesarJ}
  \label{back:fig:caesarJExpressions}
\end{figure}
\begin{figure}[ht!]
\begin{center}
\begin{footnotesize}
\begin{verbatim}
00 import eval.Eval;
01 import printPostfix.PrintPostfix;
02 import printInfix.PrintInfix;
03 import printPrefix.PrintPrefix;
04 public cclass EvalInfix extends PrintPrefix & Eval {}
\end{verbatim}
\end{footnotesize}
\end{center}
\caption{Configuraci�n que incluye las operaciones de evaluar e imprimir en formato infijo con CaesarJ}
\label{back:fig:codCaesarJ}
\end{figure}

Para realizar configuraciones, �nicamente se tiene que crear una nueva clase que extienda a las caracter�sticas que se deseen.A modo de ejemplo,la figura \ref{back:fig:codCaesarJ} muestra el c�digo necesario para crear una nueva configuraci�n que contenga las operaciones de impresi�n en formato posfijo y evaluaci�n de una expresi�n.Con todo esto, vemos como CaesarJ otorga un gran nivel de encapsulamiento y de reusabilidad de los componentes.


\section{Clases Parciales C\#}

%=========================================================================%
% Author: Pablo S�nchez                                                   %
% Paper: FOSD2010 (FOP Features)                                          %
% Version: 1.0                                                            %
% Date   : 2010/07/20                                                     %
%=========================================================================%

C\# partial classes allow developers to split the implementation of a class into several files, each one containing an slice of the global functionality of a class. All these slices are combined at compilation time to create a single class, containing all the functionality  specified in the partial classes.To be combined, all partial classes need to belong to same \emph{namespace}, have the same visibility and have been declared as partial by means of using the \imp{partial} keyword. A \emph{namespace} is simply used to group related classes and avoid name collisions.

What files must be included in a compilation unit is specified in C\# using a XML document which contain information about the project and it specifies which files must be included. Therefore, we can manipulate this file to include partial classes as desired and to produce composed classes with the desired functionality.

\begin{figure}
    \begin{center}
    \begin{small}
    \begin{verbatim}
File InitialModel/Gateway.cs    
--------------------------------------------------------    
00 namespace SmartHome {
01  public partial class Gateway {
02    protected List<Sensor> sensors;
03    protected List<Actuator> actuators;
04
05    public void emergence(Sensor s, double value) {...}
06    public bool changeValue(int id, double value) {...}
07  } // Gateway 
08 }// namespace

File LightMng/Gateway.cs
--------------------------------------------------------
09 namespace SmartHome
10 {
11    public partial class Gateway {
12        protected List<LightCtrl> lights;
13
14        public bool switchLight(int id) {...}
15    } // Gateway
16 } // namespace

File SmartHome.csproj
--------------------------------------------------------
17 </Project>...<ItemGroup>
18    <Compile Include="InitialModel\Gateway.cs" />
19    <Compile Include="LightMng\Gateway.cs" />
20 <!--  <Compile Include="SmartEnergyMng\Gateway.cs" />
21      <Compile Include="WindowMng\Gateway.cs" /> -->
22 ...
23 </ItemGroup></Project>
    \end{verbatim}
    \end{small}
    \caption{\imp{Gateway} implementation using partial classes}
    \label{fig:partialClass}
    \vspace{-15pt}
    \end{center}
\end{figure}

Figure~\ref{fig:partialClass} shows an example of usage of partial classes where the implementation of the \imp{Gateway} class for the \imp{InitialModel} and \imp{LightMng} feature has been split into two separate files with the same name, but placed in different directories (Figure~\ref{fig:partialClass} lines 00-08 and lines 09-16). The \imp{Gateway} class for the \imp{InitialModel} feature (Figure~\ref{fig:partialClass} lines 00-08) contains collections for sensors and actuators, as well as the \imp{emergence} and \imp{changeValue} methods (see Figure~\ref{fig:SH-DM}). The \imp{Gateway} class  for the \imp{LightMng} feature (Figure~\ref{fig:partialClass} lines 09-16) adds to the previous \imp{Gateway} class the collections for lights as well as the \imp{switchLight} method.

Figure~\ref{fig:compilation} shows a possible build file which specifies the \imp{Gateway} partial classes for the \imp{InitialModel} and the \imp{LightMng} features must be included in the compilation; but the corresponding partial classes for the \imp{WindowMng} or \imp{SmartEnergyMng} features, oppositely, must be excluded. Therefore, the compiler will produce a \imp{Gateway} class with functionality to manage lights, but not to manage windows, heaters or smart energy management.

Motivated by these results, some authors, such as Laguna et al~\cite{laguna:2007} proposed to used C\# partial classes as a mechanisms to implement feature-oriented design. Next section will evaluate strengthens and weaknesses of this approach using the elements presented in Section~\ref{sec:fopFeatures}.


\section{Sumario}

Durante este cap�tulo se han descrito los conceptos necesarios para entender la motivaci�n de este proyecto. Se ha descrito qu� es una l�nea de productos software, las t�cnicas utilizadas para su desarrollo, las limitaciones de la orientaci�n a objetos para su implementaci�n y las ventajas de los lenguajes orientados a caracter�sticas. No obstante, tal como se ha comentado, en muchas situaciones o los lenguajes de programaci�n orientados a caracter�sticas no est�n siempre lo suficientemente maduros como para ser transferidos a la industria, o las empresas no pueden permitirse el coste asociado a cambiar su lenguaje de programaci�n habitual. Por tanto, una posible soluci�n, ser�a intentar implementar l�neas de productos software usando las construcciones proporcionadas por un lenguaje de programaci�n de amplia difusi�n, tal como C\#. Por dicho motivo, en las siguientes secciones implementaremos una l�nea de productos software para hogares inteligentes usando las clases parciales de C\#. El siguiente cap�tulo describe la planificaci�n general seguida para la realizaci�n de este proyecto.

